\documentclass[margin,line]{res}

% File produced by Jeremy West
% This file may be distributed and/or modified:
%   1. under the LaTeX Project Public License and/or
%   2. under the GNU Public License.

% NOTE: you also need the (included) res.cls class file to compile this document
% Either place the .cls file in the same directory (folder) as this .tex file, or install it globally for your LaTeX distribution (search online for how to do so).

\usepackage[dvipsnames]{xcolor}
\usepackage[export]{adjustbox}
\usepackage{graphicx}
\usepackage{times}
\usepackage{setspace}                                    % Allows for custom margins, etc.
\usepackage{fullpage}                                    % Use the full page
\usepackage[latin9]{inputenc}                            % Font definition and input type
\usepackage[T1]{fontenc}                                 % Font output type
\usepackage{mmap}                                        % Allows glyphs (e.g. ff) to copy properly (ASCII)
\usepackage{textcomp}                                    % Supports many symbols such as copyright
\usepackage{color}                                       % Allow for colored text, etc.
\usepackage[]{hyperref}                                  % Allow hyperlinks (internal and external)
\hypersetup{                                             % Custom hyperlink settings
    pdffitwindow=true,                                  % Window fit to page when opened
    pdfstartview={XYZ null null 1.00},                   % Fits the zoom of the page to 100%
    pdfnewwindow=true,                                   % Links in new window
    colorlinks=true,                                     % false: boxed links; true: colored links
    linkcolor=black,                                     % Color of internal links (black is necessary for printing quality)
    citecolor=black,                                     % Color of links to bibliography
    urlcolor=[rgb]{0.31,0.16,0.52},                                      % Color of external links
    pdfauthor = {Yong Cai},
    pdfkeywords = {Economics, Northwestern},
    pdftitle = {CV, Yong Cai)},
    pdfsubject = {Curriculum Vitae},
    pdfpagemode = UseNone}

\oddsidemargin -.5in
\evensidemargin -.5in
\textwidth=6.0in
\itemsep=0in
\parsep=0in
% if using pdflatex:
\setlength{\pdfpagewidth}{\paperwidth}
\setlength{\pdfpageheight}{\paperheight}

	\addtolength{\topmargin}{-.3in}
	\addtolength{\textheight}{0.6in}

\newenvironment{list1}{
  \begin{list}{\ding{113}}{%
      \setlength{\itemsep}{.025in}
      \setlength{\parsep}{0in} \setlength{\parskip}{0in}
      \setlength{\topsep}{0in} \setlength{\partopsep}{0in}
      \setlength{\leftmargin}{0.17in}}}{\end{list}}
\newenvironment{list2}{
  \begin{list}{$\bullet$}{%
      \setlength{\itemsep}{0in}
      \setlength{\parsep}{0in} \setlength{\parskip}{0in}
      \setlength{\topsep}{0in} \setlength{\partopsep}{0in}
      \setlength{\leftmargin}{0.2in}}}{\end{list}}

\usepackage[nodayofweek]{datetime}                                    % Custom date format for date field
\newdateformat{mydate}{\monthname[\THEMONTH] \THEYEAR}   % Defining month year date format

\usepackage{fancyhdr}                                    % Used for custom page headers
\pagestyle{fancy}
\fancyhf{}
\renewcommand{\headrulewidth}{0.5pt}
\rhead{\footnotesize Yong Cai \thepage} %header at the right
\headsep = 0.5cm
% FIRST PAGE ONLY (redefine the plain pagestyle)
\fancypagestyle{plain}{
\fancyhf{}
\renewcommand{\headrulewidth}{0pt}
\headsep = 0.0cm
%\rhead{\footnotesize Last updated \today}
\rhead{\footnotesize \includegraphics[scale=.11]{Shorthand-Vertical-Purple}}
}


\begin{document}
\pretolerance=10000

~\\
~\\
\name{  {\LARGE  Yong Cai } \vspace*{.1in}}

\begin{resume}
\thispagestyle{plain} % to use first page footer

\begin{table}[h]
\footnotesize
\begin{tabular}{@{}p{0.20in}p{1.3in}p{1.4in}p{.9in}p{1.5in}}
& Placement Director:	& Professor Alessandro Pavan & (+1) 847-491-8266  & \href{mailto: alepavan@northwestern.edu}{alepavan@northwestern.edu}\\
& Placement Administrator:	& Lola Ittner &	(+1) 847-491-5213  & \href{mailto: econjobmarket@northwestern.edu}{econjobmarket@northwestern.edu}\\
 \end{tabular}
\end{table}
\vspace*{.05in}

\section{ Contact Information}
\vspace{.05in}
\begin{tabular}{@{}p{0.20in}p{2.5in}p{2.75in}}
 & Department of Economics           &Mobile: (+1) 773-796-0405 \\
 & Northwestern University   & \href{mailto: yongcai2023@u.northwestern.edu}{yongcai2023@u.northwestern.edu} \\
 & 2211 Campus Drive & \href{http://sites.northwestern.edu/ycz9493}{www.sites.northwestern.edu/ycz9493/}\\
 & Evanston, IL 60208  & Citizenship: Singaporean 
\end{tabular}
\vspace*{.05in}

\section{ Fields}
\begin{list1}
\item[] Econometrics
\end{list1}
\vspace*{.05in}

\section{Education}
\begin{list1}
\item[] \textbf{Northwestern University} 
\begin{list2}
		\item[] Ph.D. Economics \hfill 2023 (expected) \\[-2ex]
\end{list2} 
\item[] \textbf{London School of Economics and Political Science}
\begin{list2}
		\item[] M.Sc. Econometrics and Mathematical Economics (Distinction) \hfill 2017 
		\item[] B.Sc. Econometrics and Mathematical Economics (First Class Honours) \hfill 2016
\end{list2}
\end{list1}
\vspace*{.05in}

\section{Fellowships \& Awards}
\begin{list1}
\item[] Dissertation University Fellowship \hfill 2022--23
\item[] Distinguished Teaching Assistant Award \hfill 2021
%\item[] Northwestern University Fellowship \hfill 2017--18
\item[] Economics Department Prize  \hfill 2016
%\item[] Allyn Young Prize for Microeconomics  \hfill 2015
\item[] Economics Examiners' Prize  \hfill 2014
\item[] Singapore LSE Trust Study Award \hfill 2013--16
\end{list1}
\vspace*{.05in}

\section{Teaching Experience}
\begin{list1}
\item[] \textbf{Teaching Assistant, Northwestern University} 
	\begin{list2}
		\item[] ECON 201 Introduction to Macroeconomics \hfill 2018--21
		\item[] ECON 381-1 Econometrics  \hfill 2021
		\item[] ECON 480-3 Econometrics  \hfill 2019--20 \\[-2ex]
	\end{list2}
\item[] \textbf{Course Instructor, Northwestern University}
	\begin{list2}
		\item[] Econometrics Review for Incoming Ph.D. Students \hfill 2019 \\[-2ex]
	\end{list2}
\item[] \textbf{Teaching Assistant, London School of Economics}
	\begin{list2}
		\item[] EC210 Intermediate Macroeconomics \hfill 2016--17
	\end{list2}	
\end{list1}
\vspace*{.05in}


\section{Research Experience}
\begin{list1}
\item[] \textbf{Research Assistant, Northwestern University} 
\begin{list2}
	\item[] Professor Eric Auerbach \hfill 2020--21
	\item[] Professor Ivan Canay \hfill 2017--18 \\[-2ex]
\end{list2}  
\item[] \textbf{Research Assistant, London School of Economics} 
\begin{list2}
	\item[] Professor Shengxing Zhang \hfill 2017
	\item[] Professor David Baqaee \hfill 2016
	\item[] Behavioural Research Lab \hfill 2014--17
\end{list2}  
\end{list1}
\vspace*{.05in}

%\section{ Conferences}
%\begin{list1}
%\item[] XX
%\end{list1}

\section{Refereeing}
\begin{list1}
\item[] Journal of Business and Economic Statistics
\end{list1}
\vspace*{.05in}

\newpage

\section{Job Market Paper}
\begin{list1}
\item[] ``\href{https://yong-cai.github.io/assets/images/eigenReg.pdf}{Linear Regression with Network Centrality Measures}''
	\begin{list2}
		\item[] This paper studies the properties of linear regression on centrality measures when network data is sparse -- that is, when there are many more agents than links per agent -- and when they are measured with error. We make three contributions in this setting: (i) We characterize the amount of sparsity at which OLS estimators become inconsistent, with and without measurement error, finding that this threshold depends on the centrality measure used. Specifically, regression on eigenvector is less robust to sparsity than on degree and diffusion. (ii) We develop distributional theory for OLS estimators under measurement error and sparsity, finding that OLS estimators are subject to asymptotic bias even when they are consistent. Moreover, bias can be large relative to their variances, so that bias correction is necessary for inference. (iii) We propose novel bias correction and inference methods for OLS with sparse noisy networks. Simulation evidence suggests that our theory and methods perform well, particularly in settings where the usual OLS estimators and heteroskedasticity-consistent/robust $t$-tests are deficient. Finally, we demonstrate the utility of our results in an application inspired by De Weerdt and Dercon (2006), in which we consider consumption smoothing and social insurance in Nyakatoke, Tanzania.
	\end{list2}
\end{list1}

%\section{Working Papers (abstracts below)}
%\begin{list1}
%\item[] ``On the Implementation of Approximate Randomization Tests in Linear Models with a Small Number of Clusters'' with Ivan Canay, Deborah Kim and Azeem Shaikh. \emph{Accepted, Journal of Econometric Methods.}\\[-2ex]
%
%\item[] ``A Modified Randomization Test for the Level of Clustering'' \emph{R\&R, Journal of Business and Economic Statistics} \\[-2ex]
%
%\item[] ``On the Performance of the Neyman Allocation with Small Pilots'' with Ahnaf Rafi. \\[-2ex]
%	
%\item[] ``Heterogeneous Treatment Effects for Networks, Panels, and other Outcome Matrices'' with Eric Auerbach. \\[-2ex]
%
%\item[] ``It's not always about the money, sometimes it's about sending a message: Evidence of Informational Content in Monetary Policy Announcements'' with Santiago Camara and Nicholas Capel. \\[-2ex]
%
%\item[] ``Panel Data with Unknown Clusters'' \\[-2ex]
%
%\item[] ``Some Finite Sample Properties of the Sign Test''
%
%\end{list1}

\section{Publications}
\begin{list1}
\item[] ``\href{https://www.degruyter.com/document/doi/10.1515/jem-2021-0030/html}{On the Implementation of Approximate Randomization Tests in Linear Models with a Small Number of Clusters}'' \\
with Ivan Canay, Deborah Kim and Azeem Shaikh.
	\begin{list2}
		\item[] \emph{Accepted, Journal of Econometric Methods.}
		\item[] This paper provides a user's guide to the general theory of approximate randomization tests developed in Canay et al. (2017a) when specialized to linear regressions with clustered data. An important feature of the methodology is that it applies to settings in which the number of clusters is small -- even as small as five. We provide a step-by-step algorithmic description of how to implement the test and construct confidence intervals for the parameter of interest. In doing so, we additionally present three novel results concerning the methodology: we show that the method admits an equivalent implementation based on weighted scores; we show the test and confidence intervals are invariant to whether the test statistic is studentized or not; and we prove convexity of the confidence intervals for scalar parameters. We also articulate the main requirements underlying the test, emphasizing in particular common pitfalls that researchers may encounter. 
	\end{list2}
\end{list1}


\section{Working Papers}
\begin{list1}
\item[] ``\href{https://arxiv.org/pdf/2106.05503v3.pdf}{A Modified Randomization Test for the Level of Clustering}''
	\begin{list2}
		\item[] \emph{R\&R, Journal of Business and Economic Statistics.}
		\item[] Suppose a researcher observes individuals within a county within a state. Given concerns about correlation across individuals, at which level should they cluster their observations for inference? This paper proposes a modified randomization test as a robustness check for their chosen specification in a linear regression setting. Existing tests require either the number of states or number of counties to be large. Our method is designed for settings with few states and few counties. While the method is conservative, it has competitive power in settings that may be relevant to empirical work. \\
	\end{list2}
\item[] ``\href{https://arxiv.org/pdf/2206.04643.pdf}{On the Performance of the Neyman Allocation with Small Pilots}'' with Ahnaf Rafi. \emph{(Submitted)}
	\begin{list2}
		\item[] The Neyman Allocation and its conditional counterpart are used in many papers on experiment design, which typically assume that researchers have access to large pilot studies. This may not be realistic. To understand the properties of the Neyman Allocation with small pilots, we study its behavior in a novel asymptotic framework for two-wave experiments in which the pilot size is assumed to be fixed while the main wave sample size grows. Our analysis shows that the Neyman Allocation can lead to estimates of the ATE with higher asymptotic variance than with (non-adaptive) balanced randomization, particularly when the population is relatively homoskedastic. We also provide a series of empirical examples showing that the Neyman Allocation may perform poorly for values of homoskedasticity that are relevant for researchers. Our results suggest caution when employing experiment design methods involving the Neyman Allocation estimated from a small pilot study. \\
	\end{list2}
	
	\newpage
	
\item[] ``\href{https://arxiv.org/pdf/2205.01246.pdf}{Heterogeneous Treatment Effects for Networks, Panels, and other Outcome Matrices}'' with Eric Auerbach. \emph{(Submitted)}
	\begin{list2}
		\item[] We are interested in the distribution of treatment effects for an experiment where units are randomized to treatment but outcomes are measured for pairs of units. For example, we might measure risk sharing links between households enrolled in a microfinance program, employment relationships between workers and firms exposed to a trade shock, or bids from bidders to items assigned to an auction format. Such a double randomized experimental design may be appropriate when there are social interactions, market externalities, or other spillovers across units assigned to the same treatment. Or it may describe a natural or quasi experiment given to the researcher. In this paper, we propose a new empirical strategy based on comparing the eigenvalues of the outcome matrices associated with each treatment. Our proposal is based on a new matrix analog of the Fréchet-Hoeffding bounds that play a key role in the standard theory. We first use this result to bound the distribution of treatment effects. We then propose a new matrix analog of quantile treatment effects based on the difference in the eigenvalues. We call this analog spectral treatment effects. \\		
	\end{list2}	
	
\item[] ``\href{https://arxiv.org/pdf/2111.06365}{It's not always about the money, sometimes it's about sending a message: Evidence of Informational Content in Monetary Policy Announcements}'' with Santiago Camara and Nicholas Capel.
	\begin{list2}
		\item[] This paper introduces a transparent framework to identify the informational content of FOMC announcements. We do so by modelling the expectations of the FOMC and private sector agents using state of the art computational linguistic tools on both FOMC statements and New York Times articles. We identify the informational content of FOMC announcements as the projection of high frequency movements in financial assets onto differences in expectations. Our recovered series is intuitively reasonable and shows that information disclosure has a significant impact on the yields of short-term government bonds. \\
	\end{list2}	
\item[] ``\href{https://arxiv.org/pdf/2106.05503.pdf}{Panel Data with Unknown Clusters}''
	\begin{list2}
		\item[] Clustered standard errors and approximate randomization tests are popular inference methods that allow for dependence within observations. However, they require researchers to know the cluster structure ex ante. We propose a procedure to help researchers discover clusters in panel data. Our method is based on thresholding an estimated long-run variance-covariance matrix and requires the panel to be large in the time dimension, but imposes no lower bound on the number of units. We show that our procedure recovers the true clusters with high probability with no assumptions on the cluster structure. The estimated clusters are independently of interest, but they can also be used in the approximate randomization tests or with conventional cluster-robust covariance estimators. The resulting procedures control size and have good power. \\
	\end{list2}
\item[] ``\href{https://arxiv.org/pdf/2103.01412.pdf}{Some Finite Sample Properties of the Sign Test}''
	\begin{list2}
		\item[] This paper contains two finite-sample results about the sign test. First, we show that the sign test is unbiased against two-sided alternatives even when observations are not identically distributed. Second, we provide simple theoretical counterexamples to show that correlation that is unaccounted for leads to size distortion and over-rejection. Our results have implication for practitioners, who are increasingly employing randomization tests for inference.\\
	\end{list2}
\end{list1}


\section{Languages}
\begin{list1}
\item[] English (native), Mandarin Chinese (fluent)
\end{list1}

\section{Programming}
\begin{list1}
\item[] R, Python, Matlab, Stata
\end{list1}

\newpage

\section{ References}
\vspace{.05in}
%Pick from:
\begin{tabular}{@{}p{0.20in}p{2.75in}p{2.75in}}
 & Professor Ivan Canay          & Professor Joel Horowitz \\
 & Department of Economics   & Department of Economics \\
 & Northwestern University   & Northwestern University \\
 & 2211 Campus Drive  & 2211 Campus Drive  \\
 & Evanston, IL 60208  & Evanston, IL 60208\\
 & (+1) 847-491-2929  & (+1) 847-491-8253\\
 &  \href{mailto: iacanay@northwestern.edu}{iacanay@northwestern.edu} &  \href{mailto: joel-horowitz@northwestern.edu}{joel-horowitz@northwestern.edu} \\
\end{tabular}
\vspace{.1in} ~\\
\begin{tabular}{@{}p{0.20in}p{2.75in}p{2.75in}}
 & Professor Eric Auerbach            \\
 & Department of Economics   \\
 & Northwestern University    \\
 & 2211 Campus Drive    \\
 & Evanston, IL 60208  \\
 & (+1) 847-491-4416  \\
 &  \href{mailto: eric.auerbach@northwestern.edu}{eric.auerbach@northwestern.edu}  \\
\end{tabular}

%\vspace{20mm}
\vfill
\begin{flushright}
\small \emph{Updated \mydate\today .}
\end{flushright}



\end{resume}
\end{document}
